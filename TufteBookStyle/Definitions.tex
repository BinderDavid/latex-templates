%-------------------------------------------------------------------------------
%----Usepackages----------------------------------------------------------------
%-------------------------------------------------------------------------------
\usepackage[utf8]{inputenc}
\usepackage[german]{babel}
\usepackage[T1]{fontenc}
\usepackage{amsmath}
\usepackage{amsfonts}
\usepackage{amssymb}
\usepackage{amsthm}
\usepackage{makeidx}
\usepackage{graphicx}
\usepackage{lipsum}
\usepackage{csquotes}
\usepackage{listings}
\usepackage[style=numeric, backend=biber]{biblatex}
\usepackage{textcomp}
\usepackage{makeidx}

% Doing calculations with LaTeX units -- needed for the vertical line in the footer
% CTAN: http://www.ctan.org/pkg/calc
\usepackage{calc}

% Extended graphics support 
% There is also a package named 'graphics' - watch out!
% CTAN: http://www.ctan.org/pkg/graphicx
\usepackage{graphicx}

% Extendes support for floating objects (tables, figures), adds the [H] placing option (\begin{figure}[H]) which palces it "Here" (without any doubt).
% CTAN: http://www.ctan.org/pkg/float
\usepackage{float}

% Extended color support
% I use the command \definecolor for example. 
% Option 'Table': Load the colortbl package, in order to use the tools for coloring rows, columns, and cells within tables.
% CTAN: http://www.ctan.org/pkg/xcolor
\usepackage[table]{xcolor} 

% Nice tables
% CTAN: http://www.ctan.org/pkg/booktabs
\usepackage{booktabs}

% Better support for ragged left and right. Provides the commands \RaggedRight and \RaggedLeft. 
% Standard LaTeX commands are \raggedright and \raggedleft
% http://www.ctan.org/pkg/ragged2e
\usepackage{ragged2e}

% Create function plots directly in LaTeX
% CTAN: http://www.ctan.org/pkg/pgfplots
\usepackage{pgfplots}
\pgfplotsset{compat=1.11}
\usepackage{tikz}
\usetikzlibrary{cd}
% Special KOMA-Script package - I added it because I also use the float package in this template, see: 
% http://tex.stackexchange.com/questions/51867/koma-warning-about-toc
% CTAN: http://www.ctan.org/tex-archive/macros/latex/contrib/koma-script/doc
\usepackage{scrhack}

% Better support for marginnotes
% new command: \marginnote
% LaTeX standard command: \marginpar
% CTAN: http://www.ctan.org/pkg/marginnote
\usepackage{marginnote}

% Extended header and footer support
% CTAN: http://www.ctan.org/pkg/scrpage2
\usepackage[%
  	automark
  	,ilines
	,headsepline
	,footsepline
]{scrpage2}
% User friendly interface to change layout parameters
% CTAN: http://www.ctan.org/pkg/geometry
\usepackage{geometry}
\geometry{% siehe geometry.pdf (Figure 1)
	bottom=30mm,
	showframe=false, % For debugging: try true and see the layout frames
	margin=30mm,
	marginparsep=3mm,
	marginparwidth=20mm
}

%-------------------------------------------------------------------------------
%----Latex Math stuff-----------------------------------------------------------
%-------------------------------------------------------------------------------
\newcommand{\inv}{^{-1}}
\newcommand{\C}{\mathcal{C}}
\newcommand{\D}{\mathcal{D}}
\newcommand{\Obj}[1]{\mathtt{Obj}(#1)}
\newcommand{\Hom}[3]{\mathtt{Hom}_{#1}(#2,#3)}
\newcommand{\id}[1]{\mathbf{1}_{#1}}
%Categories
\newcommand{\Set}{\mathbf{Set}}
\newcommand{\Mon}{\mathbf{Mon}}
\newcommand{\Grp}{\mathbf{Grp}}
\newcommand{\Hask}{\mathbf{Hask}}
%-------------------------------------------------------------------------------
%----Latex stuff for Haskell listings-------------------------------------------
%-------------------------------------------------------------------------------

\definecolor{gray_ulisses}{gray}{0.55}
\definecolor{castanho_ulisses}{rgb}{0.71,0.33,0.14}
\definecolor{preto_ulisses}{rgb}{0.41,0.20,0.04}
\definecolor{green_ulises}{rgb}{0.2,0.75,0}
\lstdefinelanguage{HaskellUlisses} {
	basicstyle=\ttfamily,
	sensitive=true,
	morecomment=[l][\color{gray_ulisses}\ttfamily]{--},
	morecomment=[s][\color{gray_ulisses}\ttfamily]{\{-}{-\}},
	morestring=[b]",
	stringstyle=\color{red},
	showstringspaces=false,
	numberstyle=\tiny,
	numberblanklines=true,
	showspaces=false,
	breaklines=true,
	showtabs=false,
	emph=
	{[1]
		FilePath,IOError,abs,acos,acosh,all,and,any,appendFile,approxRational,asTypeOf,asin,
		asinh,atan,atan2,atanh,basicIORun,break,catch,ceiling,chr,compare,concat,concatMap,
		const,cos,cosh,curry,cycle,decodeFloat,denominator,digitToInt,div,divMod,drop,
		dropWhile,either,elem,encodeFloat,enumFrom,enumFromThen,enumFromThenTo,enumFromTo,
		error,even,exp,exponent,fail,filter,flip,floatDigits,floatRadix,floatRange,floor,
		fmap,foldl,foldl1,foldr,foldr1,fromDouble,fromEnum,fromInt,fromInteger,fromIntegral,
		fromRational,fst,gcd,getChar,getContents,getLine,head,id,inRange,index,init,intToDigit,
		interact,ioError,isAlpha,isAlphaNum,isAscii,isControl,isDenormalized,isDigit,isHexDigit,
		isIEEE,isInfinite,isLower,isNaN,isNegativeZero,isOctDigit,isPrint,isSpace,isUpper,iterate,
		last,lcm,length,lex,lexDigits,lexLitChar,lines,log,logBase,lookup,map,mapM,mapM_,max,
		maxBound,maximum,maybe,min,minBound,minimum,mod,negate,not,notElem,null,numerator,odd,
		or,ord,otherwise,pi,pred,primExitWith,print,product,properFraction,putChar,putStr,putStrLn,quot,
		quotRem,range,rangeSize,read,readDec,readFile,readFloat,readHex,readIO,readInt,readList,readLitChar,
		readLn,readOct,readParen,readSigned,reads,readsPrec,realToFrac,recip,rem,repeat,replicate,return,
		reverse,round,scaleFloat,scanl,scanl1,scanr,scanr1,seq,sequence,sequence_,show,showChar,showInt,
		showList,showLitChar,showParen,showSigned,showString,shows,showsPrec,significand,signum,sin,
		sinh,snd,span,splitAt,sqrt,subtract,succ,sum,tail,take,takeWhile,tan,tanh,threadToIOResult,toEnum,
		toInt,toInteger,toLower,toRational,toUpper,truncate,uncurry,undefined,unlines,until,unwords,unzip,
		unzip3,userError,words,writeFile,zip,zip3,zipWith,zipWith3,listArray,doParse
	},
	emphstyle={[1]\color{blue}},
	emph=
	{[2]
		Bool,Char,Double,Either,Float,IO,Integer,Int,Maybe,Ordering,Rational,Ratio,ReadS,ShowS,String,
		Word8,InPacket
	},
	emphstyle={[2]\color{castanho_ulisses}},
	emph=
	{[3]
		case,class,data,deriving,do,else,if,import,in,infixl,infixr,instance,let,
		module,of,primitive,then,type,where
	},
	emphstyle={[3]\color{preto_ulisses}\textbf},
	emph=
	{[4]
		quot,rem,div,mod,elem,notElem,seq
	},
	emphstyle={[4]\color{castanho_ulisses}\textbf},
	emph=
	{[5]
		EQ,False,GT,Just,LT,Left,Nothing,Right,True,Show,Eq,Ord,Num
	},
	emphstyle={[5]\color{preto_ulisses}\textbf}
}
\lstnewenvironment{code}
{\lstset{language=HaskellUlisses}}
{\smallskip}

%-------------------------------------------------------------------------------
%----Latex stuff for Commutative Diagrams---------------------------------------
%-------------------------------------------------------------------------------
%A Hack for enabling the coexistence of the babel and tikzcd package.
\newenvironment{cd}{ \shorthandoff{"}\begin{tikzcd}}{\end{tikzcd} \shorthandon{"}}

%-------------------------------------------------------------------------------
%----Latex stuff for Definition and Theorem environments------------------------
%-------------------------------------------------------------------------------
\makeatletter
\newtheoremstyle{indented}
  {3pt}% space before
  {3pt}% space after
  {\addtolength{\@totalleftmargin}{1.5em}
   \addtolength{\linewidth}{-1.5em}
   \parshape 1 1.5em \linewidth}% body font
  {}% indent
  {\bfseries}% header font
  {.}% punctuation
  {.5em}% after theorem header
  {}% header specification (empty for default)
\makeatother

\theoremstyle{indented}
\newtheorem{Def}{Definition}
\newtheorem{Lem}{Lemma}
\newtheorem{Bsp}{Beispiel}
\newtheorem{Thm}{Theorem}
\renewcommand{\qedsymbol}{$\blacksquare$}
%-------------------------------------------------------------------------------
%----Latex stuff for Textboxes--------------------------------------------------
%-------------------------------------------------------------------------------
\usepackage[many]{tcolorbox}
\newtcolorbox{textbox}[1][]{
  breakable,
  freelance,
  title=#1,
  colback=white,
  colbacktitle=white,
  coltitle=myColorMainB,
  fonttitle=\bfseries,
  bottomrule=0pt,
  boxrule=0pt,
  colframe=white,
  overlay unbroken and first={
  \draw[myColorMainB,line width=3pt]
    ([xshift=5pt]frame.north west) -- 
    (frame.north west) -- 
    (frame.south west);
  \draw[myColorMainB,line width=3pt]
    ([xshift=-5pt]frame.north east) -- 
    (frame.north east) -- 
    (frame.south east);
  },
  overlay unbroken app={
  \draw[myColorMainB,line width=3pt,line cap=rect]
    (frame.south west) -- 
    ([xshift=5pt]frame.south west);
  \draw[myColorMainB,line width=3pt,line cap=rect]
    (frame.south east) -- 
    ([xshift=-5pt]frame.south east);
  },
  overlay middle and last={
  \draw[myColorMainB,line width=3pt]
    (frame.north west) -- 
    (frame.south west);
  \draw[myColorMainB,line width=3pt]
    (frame.north east) -- 
    (frame.south east);
  },
  overlay last app={
  \draw[myColorMainB,line width=3pt,line cap=rect]
    (frame.south west) --
    ([xshift=5pt]frame.south west);
  \draw[myColorMainB,line width=3pt,line cap=rect]
    (frame.south east) --
    ([xshift=-5pt]frame.south east);
  },
}

\definecolor[named]{myColorMainA}{RGB}{0,26,153}
\definecolor[named]{myColorMainB}{RGB}{174,49,54}
%%% File encoding is ISO-8859-1 (also known as Latin-1)
%%% You can use special characters just like ä,ü and ñ

% ##############################################
% Start: Table of Contents (TOC) Customization
% ##############################################
%

% Level for numbered captions
\setcounter{secnumdepth}{5}

% Level of chapters that appear in Table of Contents
\setcounter{tocdepth}{5} % bis wohin ins Inhaltsverzeichnis aufnehmen
% -2 no caption at all
% -1 part
% 0  chapter
% 1  section    
% 2  subsection 
% 3  subsubsection
% 4  paragraph
% 5  subparagraph

% KOMA-Script code to adjust TOC
% Applying the color 'myColorMainA' which is defined in the main file (MainFile.tex)
\makeatletter
\addtokomafont{chapterentrypagenumber}{\color{myColorMainA}}
\addtokomafont{chapterentry}{\color{myColorMainA}}
\makeatother

%
% #######################
% End: Table of Contents (TOC) Customization
% #######################

% ##############################################
% Start: Floating Object Customization
% ##############################################
%

% Extended support for catioons of figures and tables etc.
% CTAN: http://www.ctan.org/pkg/caption
\usepackage[%
	font={small},
	labelfont={bf,sf},
	format=hang, % try plain or hang
	margin=5pt,
]{caption}
%

% #######################
% End: Floating Object Customization
% #######################

% ##############################################
% Start: Headings Customization
% ##############################################
%

% KOMA-Script code to customize the headings
% Applying the color 'myColorMainA' which is defined in the main file (MainFile.tex)
\addtokomafont{chapter}{\color{myColorMainA}}
\addtokomafont{section}{\color{myColorMainA}}
\addtokomafont{subsection}{\color{myColorMainA}}
\addtokomafont{subsubsection}{\color{myColorMainA}}
\addtokomafont{paragraph}{\color{myColorMainA}}
\addtokomafont{subparagraph}{\color{myColorMainA}}

% #######################
% End: Headings Customization
% #######################
% Custom command fpr the margin notes: \myMarginnote{Your Text}
% Comment on the \lineskiplimit=-\maxdimen:
% See http://tex.stackexchange.com/questions/49072/
% Without it the line spacing of the normal text was changed (ugly).
\newcommand{\myMarginnote}[1]{%
	\marginnote{% needs marginnote package
		\ifthispageodd{\RaggedRight}{\RaggedLeft}% needs ragged2e package
		\color{myColorMainB}%
		\lineskiplimit=-\maxdimen% 
		\normalfont\sffamily\scriptsize%
		#1}%
}

% ##############################################
% Start: Header and Footer Customization
% ##############################################
%

% KOMA-Script code for header and footer font
\setkomafont{pageheadfoot}{%
	\normalfont\sffamily\bfseries
	}
\setkomafont{pagefoot}{%
	\normalfont\sffamily
	}
\setkomafont{pagenumber}{%
	\normalfont\rmfamily
	}

% Define width of header
\setheadwidth[0pt]{textwithmarginpar}

% Define with of header line
\setheadsepline{0.4pt}

% Define width of footer
\setfootwidth[0pt]{text}
% Define with of footer line (here: no line)
\setfootsepline[text]{0pt}

% Some calculations
% calc package is needed which is loaded here: 01_Preamble/CommonPackages.tex
% If you want to understand the calculations visit:
% http://en.wikibooks.org/wiki/LaTeX/Page_Layout
\newlength{\myLenghthFootAbstand}
\setlength{\myLenghthFootAbstand}{\paperheight-1in-\topmargin- \headheight-\headsep-\textheight-\footskip}
\newlength{\myLenghthTemp}
\setlength{\myLenghthTemp}{\myLenghthFootAbstand+\baselineskip}

% Define content of header and footer
% Using some scrpage2 commands here. The scrpage2 package is loaded here: 01_Preamble/KOMA-Script-Packages.tex
% Some LaTeX magic...
% Clear all defaults
\clearscrheadfoot
% Header
\ohead{%
	\textcolor{myColorMainA}{\headmark}
	}
% Left (even page numbers) footer
\lefoot%
[% scrplain style (begin)
	\setlength{\unitlength}{\myLenghthFootAbstand}%
	\begin{picture}(0,0)%
		\put(0,-1)%
		{%
			\makebox(0,0)[lb]%
			{%
				\rule{0.4pt}{\myLenghthTemp}%
			}%
		}%
	\end{picture}\llap{\pagemark~}%
]% scrplain style (end)
%
{% scrheadings style (begin)
	\setlength{\unitlength}{\myLenghthFootAbstand}%
	\begin{picture}(0,0)%
		\put(0,-1)%
		{%
			\makebox(0,0)[lb]%
			{%
				\rule{0.4pt}{\myLenghthTemp}%
			}%
		}%
	\end{picture}\llap{\pagemark~}%
}% scrheadings style (end)

% Right (odd page numbers) footer
\rofoot%
[% scrplain style (begin)
	\rlap{~\pagemark}%%
	\setlength{\unitlength}{\myLenghthFootAbstand}%
	\begin{picture}(0,0)%
		\put(0,-1)%
		{%
			\makebox(0,0)[lb]%
			{%
				\rule{0.4pt}{\myLenghthTemp}%
			}%
		}%
	\end{picture}%
]% scrplain style (end)
%
{% scrplain style (begin)
	\rlap{~\pagemark}%%
	\setlength{\unitlength}{\myLenghthFootAbstand}%
	\begin{picture}(0,0)%
		\put(0,-1)%
		{%
			\makebox(0,0)[lb]%
			{%
				\rule{0.4pt}{\myLenghthTemp}%
			}%
		}%
	\end{picture}%
}% scrplain style (end)

%
% #######################
% End: Header and Footer Customization
% #######################

% This is an suggestion from Axel Reichert (LaTeX package author)
% See CTAN: http://www.ctan.org/author/reichert
% See CTAN: http://www.ctan.org/pkg/l2tabu-english (Cgapter: 1.8 Should I use \sloppy?)

\tolerance 1414
\hbadness 1414
\emergencystretch 1.5em
\hfuzz 0.3pt
\widowpenalty=10000
\vfuzz \hfuzz
\raggedbottom