\documentclass[11pt]{beamer}
\usetheme{CambridgeUS}

\usepackage[utf8]{inputenc}
\usepackage[english]{babel}

\usepackage{amsmath}
\usepackage{amsfonts}
\usepackage{csquotes}
\usepackage{bussproofs}
\usepackage{amssymb}
\usepackage{tikz}
\usetikzlibrary{calc,positioning}
\usepackage{graphicx}
\usepackage{listings}
\usepackage{color}
\usepackage{textcomp}
\definecolor{listinggray}{gray}{0.9}
\definecolor{lbcolor}{rgb}{0.9,0.9,0.9}
\lstset{
	backgroundcolor=\color{lbcolor},
	tabsize=4,
	rulecolor=,
	language=Java,
        basicstyle=\scriptsize,
        upquote=true,
        aboveskip={1\baselineskip},
        columns=fixed,
        showstringspaces=false,
        extendedchars=true,
        breaklines=true,
        prebreak = \raisebox{0ex}[0ex][0ex]{\ensuremath{\hookleftarrow}},
        frame=single,
        showtabs=false,
        showspaces=false,
        showstringspaces=false,
        identifierstyle=\ttfamily,
        keywordstyle=\color[rgb]{0,0,1},
        commentstyle=\color[rgb]{0.133,0.545,0.133},
        stringstyle=\color[rgb]{0.627,0.126,0.941},
}
\author{David Binder}
\title{Title of Presentation}
%\setbeamercovered{transparent} 
\setbeamertemplate{navigation symbols}{} 
%\logo{} 
\institute{}
\date{} 
%\subject{}

\begin{document}
\begin{frame}
	\titlepage
\end{frame}

\begin{frame}
	\tableofcontents
\end{frame}

\section{Motivation}
\begin{frame}{Frame Title 1}
	In distinction to Benacerraf, we pursue a \emph{pragmatic} interpretation of a structuralist philosophy of mathematics.\pause
	
	\begin{block}{Tsementzis 2017}
		 \enquote{We take the practical reading of the structuralist thesis to indicate a \enquote{design constraint} for a foundation of mathematics.
	 The motivating problem is thus to create a foundational system [\ldots] such that any grammatically well-formed property about a mathematical object is invariant under the appropriate criterion of identity for that object (as those are formalized in the given system).}
	\end{block}\pause
	(For \enquote{mathematical object} think group, topological space, category\ldots For \enquote{appropriate criterion of identity} think group isomorphism, homeomorphism, equivalence\ldots)
\end{frame}

\begin{frame}{Overview}
	\begin{itemize}
		\item Analysis of the problem.
			\begin{itemize}
				\item Why the relation \enquote{$\in$} is a problem. (Hint: It is untyped.)
				\item Why the relation \enquote{$=$} is a problem. (Hint: It is too strict.)
			\end{itemize}\pause
		\item Towards a solution.
			\begin{itemize}
				\item Foundational Theories
				\item Homotopy Types / $\omega$-Groupoids
				\item \enquote{Homotopy (Type Theories)} and\\
				\enquote{(Homotopy Type) Theories}
				\item Univalence
			\end{itemize}
	\end{itemize}
\end{frame}

\begin{frame}
	\begin{center}
		{\huge{Thank you for your attention.}}
	\end{center}
\end{frame}
\end{document}
